\begin{center}
\indent
\textit{Rappresentazione con matrici, diagonalizzazione.}
\end{center}

\section{Applicazione lineari: rappresentazione e diagonalizzazione}

Un'applicazione lineare \`e un'applicazione fra due spazi vettoriali $L : V \to W$ ($V$ e $W$ spazi vettoriali in $\mathbb{R}$) che conserva le operazioni:

\begin{enumerate}
    \item conserva la somma: $L (v + v') = L(v) + L(v')$. L'immagine della somma di due vettori \`e uguale alla somma delle immagini dei vettori.
    \item conserva i prodotti scalari: $L(r \cdot v) = r \cdot L(v)$. L'immagine del prodotto di un vettore per uno scalare \`e uguale al prodotto delo scalare per l'immagine del vettore.
\end{enumerate}

% TODO: esercizio

Trovare un esempio di applicazione lineare in uno spazio vettoriale visto.